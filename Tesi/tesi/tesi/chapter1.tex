\newpage
\clearpage{\pagestyle{empty}\cleardoublepage}
\chapter{Capitolo 1}
\label{intro}
\thispagestyle{plain}
\renewcommand{\labelenumi}{\theenumi.}

%\begin{quotation}
%{\footnotesize \noindent{\emph{This is a quotation chosen by the
%author to synthesize the chapter.} }
%\begin{flushright}
%Authors of quotation, \emph{Title of the Volume the Quotation Comes
%From}, $year$
%\end{flushright} }
%\end{quotation}
%\vspace{0.5cm}
%\begin{comment}
Il progetto globale
Il progetto globale a cui si far� � CANTICO - PLATFORM TO ATTRACT MORE AUDIENCE AND SPREAD OPERA AND PROSE THEATER BY USING IMMERSIVE TECHNOLOGIES. Il progetto intende favorire la crescita del pubblico dei teatri d�opera e di prosa mediante l�uso di nuove tecnologie per attrarre giovani e giovanissimi.\cite{Min}\par

L�arte italiana, ed in particolare l�opera lirica e il teatro di prosa, non sono soltanto la punta di diamante del nostro Paese, ma una risorsa economica strategica, un patrimonio da valorizzare, con produzioni e un�offerta formativa d�eccellenza riconosciute a livello internazionale. Tuttavia, in questi ultimi anni la marginalit� di questo settore del Made in Italy � in costante diminuzione. Infatti, i teatri d�opera, o meglio le Fondazioni lirico-sinfoniche che li gestiscono, attraversano da anni una profonda crisi economica che rischia di minarne alla radice non solo la qualit� ma finanche la sostenibilit�.\par

Lo stato attuale evidenzia la necessit� di migliorare le capacit� di autofinanziamento dei teatri lirici e di prosa, sia attraverso l�incremento di pubblico sia attraverso l�arricchimento dei servizi che essi erogano. In tale ottica, il progetto CANTICO intende sviluppare e sperimentare metodi e strumenti ICT per la creazione di servizi mirati a valorizzare il patrimonio delle produzioni liriche e dei teatri di prosa italiani.\par

Quattro sono le direzioni verso cui l�intero progetto intende operare:\par

\begin{enumerate}
	\item Favorire la crescita della base di pubblico che frequenta i teatri d�opera e di prosa, in particolare mediante l�uso di nuove tecnologie per attrarre i pi� giovani;\par

	\item Estendere la fruizione delle produzioni oltre il tradizionale canale della rappresentazione dal vivo, attraverso strumenti di fruizione remota innovativi, sia in tempo reale sia on-demand, e attraverso l�adozione di tecniche e metodologie per favorire l�accessibilit� ai disabili;\par

	\item Accompagnare la crescita di un pubblico appassionato e consapevole e formare gli artisti del domani, attraverso azioni di sensibilizzazione e formazione rivolte a studenti di ogni ordine e grado basate su tecnologie innovative, e attraverso l�adozione di soluzioni in grado di favorire una didattica inclusiva;\par

	\item Migliorare la capacit� di programmazione dei teatri lirici e di prosa attraverso strumenti innovativi di raccolta ed analisi dei dati.
\end{enumerate}\par

L�idea � di dare vita ad una piattaforma multicanale capace di comunicare in maniera chiara, efficace e coinvolgente. La piattaforma dovr� supportare i servizi tradizionali e dovr� essere aperta all�integrazione con servizi di terza-parte, in una logica di sussidiariet� e cooperazione finalizzati alla creazione di una esperienza user-centered. Si vuole inoltre sviluppare una piattaforma che dovr� consentire la creazione di una community in cui condividere informazioni e impressioni su un evento, fino a creare dei veri e propri $``$percorsi esperienziali$"$ .\par


\vspace{\baselineskip}
In particolare questo elaborato si occuper� del primo obiettivo inserito nel progetto:\par

\setlength{\parskip}{0.0pt}
\begin{itemize}
	\item Individuazione e studio di metodi e strumenti di personalizzazione e condivisione dell'esperienza utente. In particolare si vuole completare un�attivit� di ricerca applicata al profiling sfruttando tecniche di sentiment analysis e di emotion mining. L�applicazione di tali tecniche per l�estrazione di informazioni implicite, consentiranno di identificare gli interessi di un utente, ovvero gli interessi che ha in comune con altri utenti, anche in un evento specifico, e potr� aiutare gli organizzatori dell�evento ad attrarre pi� visitatori ad eventi simili in futuro e a migliorare i servizi e le esperienze future.\par
